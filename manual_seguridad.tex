\documentclass{book}
\usepackage[a4paper, total={18cm,20cm},top=2cm,left=2cm]{geometry}
\usepackage[spanish]{babel}
\title{Manual de Seguridad}
\author{Juan José González Ramírez}
\begin{document}
    \maketitle
    \tableofcontents
    \chapter{Introducción}
        Los requerimientos de seguridad que involucran las tecnologías de la
        información, en pocos años han cobrado un gran auge y más aun con las de carácter
        globalizador como lo son la de internet y en particular, la relacionada con el Web,
        situación que ha llevado a la aparición de nuevas amenazas a los sistemas
        computarizados.

        Es necesario normalizar el uso adecuado de estas destrezas tecnológicas para
        aprovechar estas ventajas, evitar su uso indebido en los bienes, servicios
        de la organización.

        Estos lineamientos de seguridad para los equipos y programas
        de informática, aparecen como el instrumento de apoyo a los servidores, acerca de la
        importancia de la información, protección de servicios críticos.

        Deberán seguir un proceso de verificación, actualización cada dos
        años o cuando sea necesario sujetos a los cambios organizacionales relevantes:
        crecimiento de la planta personal, cambio en la infraestructura informática, desarrollo
        de nuevos servicios, entre otros.
    \
    \chapter{Objetivos}
        \section{General}
            Establecer lineamientos de trabajo para la Unidad de Informática con el fin
            seguir los procedimientos adecuados para proporcionar seguridad en el manejo,
            resguardo de información e infraestructura.
        \
        \section{Específicos}
            Dar a conocer a cada técnico sobre los procedimientos y normativas a seguir en
            la UI.

            Mostrar a las demás unidades los lineamientos que deben seguir
            para la gestión de software, hardware.
        \
        \section{Alcance}
            Estos lineamientos aplican la gestión de software, hardware,
            para el manejo de la información crítica de la institución,
            desde los accesos a datos hasta la eliminación de los mismos, instalaciones de
            equipos, servicios, mantenimientos.
        \
    \

    \chapter{Definiciones}
        \textbf{Área Crítica}: Es el área física donde se encuentra instalado el equipo de
        informática y telecomunicaciones que requiere de cuidados especiales y que son
        indispensables para el funcionamiento continuo de los sistemas de comunicación a los
        están conectados.

        \textbf{Auditoria}: Llevar a cabo una inspección y examen independiente de los registros
        del sistema y actividades para probar la eficiencia de la seguridad de datos y
        procedimientos de integridad de datos, para asegurar el cumplimiento con la política
        establecida y procedimientos operativos y para recomendar cualquier cambio que se
        estime necesario.

        \textbf{Backup (copia de seguridad)}: Los programas y/o técnicas de respaldo (backup),
        son las que permiten realizar una copia espejo de la información alojada en una base de
        datos, servidor y/o computadora personal, almacenándola en un dispositivo de
        almacenamiento masivo como disco duro externo u otro dispositivo de red destinado
        para este fin, con el objetivo de realizar la recuperación de la información y evitar
        pérdidas de información críticas.

        \textbf{Bases de Datos (Database)}: Conjunto de datos relacionados que se almacenan de
        forma que se pueda acceder a ellos de manera sencilla, con la posibilidad de
        relacionarlos y ordenarlos en base a diferentes criterios, etc. Las bases de datos son uno
        de los grupos de aplicaciones de productividad personal más extendidos.

        \textbf{Control de acceso}: Técnica usada para definir el uso de programas o limitar la
        obtención y almacenamiento de datos a una memoria. Una característica o técnica en un
        sistema de comunicaciones que permite o niega el uso de algunos componentes o
        algunas de sus funciones.

        \textbf{Equipo de telecomunicaciones}: Todo dispositivo capaz de transmitir y/o recibir
        señales digitales o analógicas para comunicación de voz, datos y video, ya sea
        individualmente o de forma conjunta.

        \textbf{Equipo de informática}: Dispositivo con la capacidad de aceptar y procesar
        información con base a programas establecidos o instrucciones previas, teniendo la
        oportunidad de conectarse a una red de equipos o computadoras para compartir datos y
        recursos, entregando resultados mediante despliegues visuales, impresos o audibles.

        \textbf{Enrutador}: Es un dispositivo que proporciona conectividad a nivel de red. Su
        función principal consiste en enviar o encaminar paquetes de datos de una red a otra, es
        decir, interconectar subredes, entendiendo por subred un conjunto de máquinas IP que
        se pueden comunicar sin la intervención de un enrutador y que por tanto tiene prefijos
        de red distintos.

        \textbf{Firewall (cortafuegos)}: Sistema colocado entre una red local e internet que asegura
        dicha red local y mantiene a los usuarios no autorizados fuera de la misma. Firewall en
        un sistema operativo, es un programa especializado en resguardar la información que se
        aloja en la computadora local de cualquier acceso remoto de tipo malicioso o ataques
        informáticos.

        \textbf{Internet}: Es una convergencia de conceptos computacionales para presentar y
        enlazar información que se encuentra dispersa a través de páginas Web en una forma
        fácilmente accesible.

        \textbf{Programas Freeware}: Programas que se ofrecen al público sin ningún costo, pero
        que mantiene un copyright sobre ellos. Es decir, se pueden usar sin problemas, pero no
        se pueden utilizar como parte de otros programas o modificarlos de ninguna manera.

        \textbf{Red informática}: Conjunto de ordenadores conectados directamente por cable,
        remotamente vía conmutadores de paquetes, o por otro procedimiento de comunicación.

        \textbf{Software de distribución gratuita}: Programas que se distribuyen a través de
        internet de forma gratuita.

        \textbf{Servidor:} Genéricamente, dispositivos de un sistema que resuelve las peticiones
        de otros elementos del sistema, denominados clientes. Computadora conectada a una
        red que pone sus recursos a disposición del resto de los integrantes de la red. Suele
        utilizarse para mantener datos centralizados o para gestionar recursos compartidos.
        Internet es en ultimo termino, un conjunto de servidores que proporcionan servicios de
        transferencias de ficheros, correo electrónico o páginas Web, entre otros.
    \
    \chapter{Generales}
        Cada una de las unidades organizativas de la organización, deberá elaborar los
        planes de contingencia que correspondan a las actividades críticas que realicen a través
        de los sistemas de información.

        Los presentes lineamientos deberán ser divulgados por la unidad de informática
        a través de la gerencia, a todo el personal involucrado que utilice equipos y programas
        informáticos.

        La unidad de informática, podrá acceder a la información de un servidor público
        cuando se presuma alguna falta grave que amerite una sanción.

        Cuando a un servidor público se le esté realizando un proceso de investigación
        por alguna falta o negligencia y este, para realizas sus funciones, requiera tener acceso a
        la información y a las operaciones que se realicen, la Unidad de Informática podrá
        restringirles el acceso a los equipos informáticos de la dirección.
        \
    \chapter{Lineamientos de Seguridad}
        La unidad de informática es la responsable de brindar servicio directo al usuario,
        en lo que respecta al equipamiento, instalación, actualización, cambio de lugar y
        programación informática, a fin de permitirle el uso de los equipos, de la infraestructura
        de red y servicios asociados a ellos, en forma eficaz y eficiente.

        \section{Del equipo de la instalación de equipos informática}
            Todo el equipo de informática (computadoras, estaciones de trabajo, accesorios
            y partes), que esté conectado a la red de la dirección aquel que en forma autónoma se
            tenga y sea propiedad de la dirección, deberá de sujetarse a las configuraciones de la red
            que se conectan.
            La protección física y uso adecuado de los equipos es responsabilidad de quienes
            en un principio se les asigna.
        \

        \section{De la atención de fallas o problemas de hardware, software}
            La atención de fallas o problemas menores de los equipos informáticos, se hará
            por vía telefónica o correo electrónico (atención inmediata y una solución a la falla
            detectada en hardware y software). Para los casos en que se tuviese que solicitar
            servicios de reparación o cambios de partes en los equipos y/o actualizaciones en
            hardware o software, se solicitara por escrito a la Unidad de Informática.

            La unidad de informática atenderá las fallas o problemas de los equipos
            informáticos que estén o no con mantenimiento preventivo o correctivo, documentará la
            causa de la falla, emitirá un diagnóstico y dará las recomendaciones a los usuarios de las
            posibles soluciones para la rehabilitación o reparación de los equipos.

            La atención de fallas o problemas podrá realizarse en el lugar de trabajo
            dependiendo del problema presentado y de la factibilidad para solucionarlo.
        \

        \section{Del mantenimiento de equipo de informática}
            La unidad de informática, coordinará y verificará los servicios de
            mantenimiento preventivo y correctivo para los equipos de informática, propiedad de la
            organización, sean ejecutados según el contrato de servicio, por parte de
            la empresa contratada para que proporcione estos servicios.
        \

        \section{De la actualización del equipo}
            La unidad de informática, es la responsable de proponer a las máximas
            autoridades de la actualización de los equipos de informática y red.
            Todo agregado o adhesión de repuestos incremente la vida útil del equipo
            informático, debe de ser reportado al encargado de bienes muebles.
        \

        \section{De la reubicación del equipo de informática}
            En caso de existir, en las áreas organizativas, personal con conocimientos
            básicos en el área informática, la jefatura o dirección correspondiente deberá informar a
            la unidad de informática, los cambios o reubicaciones tanto de hardware como del
            software por escrito, en dicho documento se deberá agregar el nombre del encargado
            que realizo el movimiento.

            En caso de no existir dicho personal, la jefatura o dirección correspondiente
            deberá solicitar a la unidad informática, la realización de los cambios o reubicaciones
            tanto de hardware como del software.

            Se deberá notificar a la unidad de informática, de los cambios o sustituciones del
            equipo inventariado (cambio de monitores, de impresores etc. Entre ambientes que se
            realicen), así como también si se cambiara de responsable del equipo.
        \

        \section{Del control de accesos}
            \subsection{Del control a las áreas críticas}
                La unidad de información, identificara las áreas críticas o de acceso restringido
                para el personal.
                El jefe del área organizativa correspondiente, será el responsable de autorizar o
                no el ingreso de personal a las áreas consideradas críticas.
            \

            \subsection{Del control de acceso al equipo de informática}
                El servidor público a quien se le asigne un equipo será el responsable de su
                custodia y buen uso del mismo y responderá a su extravió o daño no justificable.

                Dada la naturaleza de los sistemas operativos y su conectividad en red, la
                Unidad de Informática, tiene la facultad de acceder a cualquier equipo de informática
                que esté conectado en la red aun cuando no esté bajo la responsabilidad de la sección.
            \

            \subsection{Del control de acceso local a la red}
                La Unidad de Informática, es la responsable de proporcionar a los usuarios el
                acceso a los recursos informáticos que estén o no en red.

                Dado el carácter unipersonal del acceso a la red, la unidad informática,
                verificara, a través de visitas periódicas a los usuarios, el uso responsable de los
                recursos informáticos que se comparten en la red.

                El acceso a equipo especializado en informática (servidores, enrutadores, bases
                de datos, etc.) conectado a la red será administrado por la Unidad Informática.
                Todo el equipo de informática que este o sea conectado a la red, o aquellos que
                en forma autónoma se tengan y que sean propiedad de la organización, deberán sujetarse a
                los procedimientos de acceso que emita la Unidad de Informática.
            \

            \subsection{Del control de acceso remoto}
                La Unidad de Informática, será la responsable de proporcionar los servicios de
                acceso a los recursos informáticos, disponibles a través de la red.
                Los usuarios de estos servicios deberán sujetarse a las configuraciones y
                especificaciones ya instaladas y no podrán hacer cambios a éstas.
            \

            \subsection{Del acceso a los sistemas de información}
                La instalación y uso de los sistemas de información se regirán por lo establecido
                en la definición de políticas y procedimientos de controles generales en los sistemas de
                información de las normas técnicas de control interno específicas de la organización.

                El control de acceso a cada sistema de información será determinado por la
                gerencia TI responsable de generar y procesar los datos, sobre la base de los
                niveles de responsabilidad asignados a cada servidor público para su lugar de trabajo.
                
                La creación de nuevas cuentas con acceso a los sistemas de información deberá
                ser solicitado vía correo electrónico o por escrito, esta solicitud deberá contener el
                nombre del servidor público, nivel de acceso a la información (solo consulta, ingreso,
                modificación de datos, etc.). unidad que solicita, firma y sello de jefe inmediato, así
                como la autorización de la gerencia general.
            \
            
            \subsection{Del acceso a Internet}
                La organización, será responsable de autorizar la información a
                publicar en el sitio Web Institucional.

                La gerencia TI, será la responsable de actualizar las veces que sea necesario hacerlo,
                la información que se publique en el sitio web institucional.
                
                Los accesos a las páginas web a través de los equipos autorizados, deben
                utilizarse responsablemente y no descargar programas o información que pueda dañar a
                los programas y equipos propiedad de la organización.
                
                Los servicios de internet estarán sujetos a la disponibilidad en la infraestructura
                de red de datos en la organización.

                El servicio de internet no deberá utilizarse para navegar por páginas con
                información obscena o para enviar correos electrónicos que dañen la integridad moral
                de las personas. El mal uso del internet será sancionado con la suspensión del servicio.
                El tiempo de la sanción será determinado de acuerdo a lo estipulado en el reglamento
                interno de trabajo de la organización.

                La unidad de informática, no se hará responsable por problemas externos de
                conectividad y comunicación por parte del proveedor del servicio, sin embargo, se
                comunicará con el proveedor para hacer el reclamo respectivo.
            \

            \subsection{De la utilización de recursos de redes}
                Corresponde a la unidad de informática el administrar, mantener y actualizar la
                infraestructura de la red de la organización.
                
                Los recursos disponibles a través de la red, serán de uso exclusivo para asuntos
                relacionados con las actividades del puesto de trabajo y del lugar donde está asignado.
            \
        \

        \section{Del Software}
            \subsection{De la adquisición de software}
                La unidad de informática, establecerá los mecanismos de sustitución de sistemas
                y programas informáticos.

                Corresponderá a la unidad de informática, el proporcionar asesoría y apoyo a
                técnico para licenciamiento, cobertura, transferencia, certificación y vigencia de los
                programas informáticos.

                La unidad de informática, será encargada de recomendar la adquisición de
                programas informáticos de vanguardia, de acuerdo a lo estipulado en los lineamientos
                sobre especificaciones técnicas para la adquisición de equipos y programas de
                informática de la organización.

                La unidad informática, será la encargada de asesor y apoyar técnicamente para
                mantener actualizado los estándares de configuración de los sistemas operativos,
                programas comerciales, base de datos y comunicación.

                Para la adquisición de nuevas licencias o actualizaciones de sistemas operativos,
                programas comerciales, base de datos y comunicaciones, equipos, accesorios y
                repuestos informáticos, será la unidad de informática quien proporcione la asesoría y
                opinión técnica, además de autorizar la adquisición de lo descrito anteriormente
            \

            \subsection{De la instalación de software}
                Corresponde a la unidad de informática, la instalación y supervisión del software
                básico para cualquier tipo de equipo informático.

                La unidad de informática, es la responsable de brindar asesoría y supervisión
                para la instalación de software informático y de telecomunicaciones.

                La unida de informática, será responsable de que, en los equipos de informática,
                de telecomunicaciones y en dispositivos basados en sistemas informáticos, únicamente
                se instalen software con licenciamiento propiedad de la organización y acorde a los
                derechos que la licencia delimite.

                La instalación y uso de paquetes y programas informáticos gratuitos (freeware) o
                sin costo alguno para la organización, será autorizado por la unidad de informática,
                respetando la ley de propiedad intelectual.

                La unidad de informática, proporcionara asesoría y apoyo técnico por medio de
                la instalación de versiones actualizadas, que ayuden a solventar problemas detectados o
                modificaciones que contribuyen a mejorar la utilización de los equipos informáticos
                para las unidades.

                Se prohíbe la instalación de programas que no posean su licencia de software,
                juegos u otro programa informático que no tengan relación con las funciones y
                actividades que el servidor público desempeñe en su lugar de trabajo. La unidad de
                informática, monitoreara su cumplimiento.

                Es prohibida la instalación de software que pudiera poner en riesgo los recursos
                o la información de la Dirección. El no cumplimiento de este numeral será sancionado
                de acuerdo al reglamento interno de trabajo vigente.

                Para proteger la integridad de los sistemas informáticos y de
                telecomunicaciones, es imprescindible que todos y cada uno de los equipos
                involucrados dispongan como mínimo de software de seguridad (antivirus, vacunas,
                firewall software, etc.), u otros.

                Que aplique y que tenga relación con las funciones y actividades que el servidor
                público desempeñe en su lugar de trabajo.

                La protección y manejo de los sistemas y programas informáticos, corresponde a
                las personas o grupos que se les asigna y les compete notificar cualquier problema de
                estos a su jefe inmediato superior, quien deberá informar a la unidad de informática para
                proporcionar una solución oportuna a dicho problema.
            \

            \subsection{De la actualización de software}
                Corresponde a la organización a través de la unidad de informática autorizar
                cualquier adquisición y actualización de software.

                La actualización del software de uso común lo llevará a cabo la unidad de
                informática, y se hará de acuerdo a las necesidades institucionales.
            \

            \subsection{De la Auditoría de software instalado}
                La unidad de informática, será la responsable de realizar la auditoria de software
                instalado.

                La unidad de informática, realizara, al menos dos veces al año, revisiones de los
                equipos informáticos, para asegurar que los programas instalados, en caso de necesitar
                licencia, cuenten con una licencia vigente.

                Los servidores públicos, cuyas computadoras cuenten con software instalado de
                versión de prueba, en caso de necesitar licencia valida, deberán presentar la necesidad y
                justificación para adquisición del software a la unidad de informática a través de la
                organización.

                Todo programa adquirido, es propiedad de la organización y mantendrá los derechos
                que la ley de propiedad Intelectual le confiera.

                Todos los programas, bases de datos, sistemas operativos, interfaces,
                desarrollados a través de los recursos de la organización, se mantendrán como propiedad de
                la Dirección respetando la propiedad intelectual del mismo.

                El software disponible en cada equipo informático es propiedad de la organización,
                quedando prohibida su distribución y reproducción.

                Los códigos fuentes de los sistemas de información, deberán estar en la unidad
                de informática para efectos de custodia y control.

                La unidad de informática, podrá verificar el registro de las licencias, sistemas y
                programas informáticos propiedad de la organización.

                Es obligación de todos los usuarios que manejen información, mantener el
                respaldo correspondiente de la misma, ya que se considerara como un activo de la
                organización que debe preservarse.

                Los datos, las bases de datos, la información generada por el personal y los
                recursos informáticos de la Dirección deberán estar debidamente resguardados.
            \

            \subsection{De la supervisión y evaluación}
                Es responsabilidad de la unidad de informática, la supervisión y evaluación de
                los sistemas de información que involucren aspectos de seguridad lógica y física, las
                cuales deberá realizarse cada año.
                
                Los sistemas de información institucional deben estar bajo monitoreo y
                actualización permanente.
            \
        \
    \
    
\end{document}